%
% This file can be rendered by calling 
% latexmk publication_list.tex
%
% NOTE: This file purposely does not allow UTF-8
% and other unicode characters. This is to catch
% character combinations (e.g. ``fi'', ``fl'') that
% do not render as (likely) intended if their unicode
% counterparts are used.
%

\documentclass[]{article}
\usepackage[utf8]{inputenc}
\usepackage[T1]{fontenc}
\usepackage[colorlinks,urlcolor=blue]{hyperref}
\usepackage[
  backend=biber,
  sorting=ydnt,
  defernumbers=true,
  doi=true,
  refsection=section
  ]{biblatex}
\AtBeginBibliography{\small}

\newcommand{\addnewbibyear}[1]{%
\begin{refsection}[../publications-#1.bib]
\nocite{*}
\printbibliography[title={Publications in #1}]
\end{refsection}
}

%opening
\title{deal.II publication list}
\author{The deal.II authors and contributors}


% We don't ever use the output of this running pdflatex on this file
% -- we only use it to find errors. As a consequence, the many
% warnings on overfull hboxes are annoying and obscuring what really
% constitutes a problem. Avoid this by just being totally relaxed
% about overfull and underfull boxes:
\hbadness=10000
\vbadness=10000
\hfuzz=10in
\vfuzz=10in

\begin{document}

\maketitle

\addnewbibyear{1998}
\addnewbibyear{1999}
\addnewbibyear{2000}
\addnewbibyear{2001}
\addnewbibyear{2002}
\addnewbibyear{2003}
\addnewbibyear{2004}
\addnewbibyear{2005}
\addnewbibyear{2006}
\addnewbibyear{2007}
\addnewbibyear{2008}
\addnewbibyear{2009}
\addnewbibyear{2010}
\addnewbibyear{2011}
\addnewbibyear{2012}
\addnewbibyear{2013}
\addnewbibyear{2014}
\addnewbibyear{2015}
\addnewbibyear{2016}
\addnewbibyear{2017}
\addnewbibyear{2018}
\addnewbibyear{2019}
\addnewbibyear{2020}
\addnewbibyear{2021}

\end{document}
